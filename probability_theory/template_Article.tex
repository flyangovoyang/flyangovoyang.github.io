\documentclass[11pt]{article}
\usepackage{xeCJK}

%opening
\title{概率论与数理统计}
\author{Yang Fu}
\date{Dec 2022}

\begin{document}

\maketitle

\begin{abstract}

\end{abstract}

\section{随机事件和概率}

\textbf{随机现象}:在个别试验中结果呈现不确定性,在大量重复试验中结果具有统计规律性的现象。

\textbf{随机试验}:研究随机现象的试验或者观察。

\textbf{随机事件}:随机试验的每一种结果。

\textbf{频率和概率}:随着试验次数增大,频率趋于稳定,此时的频率值视为概率值。

\textbf{条件概率}:已知事件A出现的条件下,事件B出现的条件概率$P(B|A)$为

\begin{equation}\label{key}
	P(B|A) = \frac{P(AB)}{P(A)}
\end{equation}

\textbf{贝叶斯公式}:设$\{H_1, ..., H_n, ...\}$是可数个概率不为0的事件构成完全事件组,且$P(A)>0$,则对于任意事件$A$,有

\begin{equation}\label{key}
	P(H_k|A) = \frac{P(AH_k)}{P(A)} = \frac{P(A|H_k)P(H_k)}{\sum_jP(A|H_j)P(H_j)} (k=1, 2, ..., n, ...)
\end{equation}
\end{document}
